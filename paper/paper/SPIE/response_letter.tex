\documentclass[10pt,letter]{article}
\usepackage[utf8]{inputenc}
\usepackage{csquotes}
\usepackage[british]{babel}
\usepackage{url}
\usepackage{graphicx}
\graphicspath{ {figures/} }
\usepackage{geometry}
\geometry{
  lmargin=2cm,
  rmargin=2cm,
  bmargin=2cm,
}
\usepackage{amsmath}
\usepackage{graphicx}
\usepackage[
    backend=biber,
    style=authoryear,
    natbib=true,
    sortlocale=en_US,
    url=false,
    doi=true,
    eprint=false
]{biblatex}
\addbibresource{report.bib}
\usepackage{fancyhdr}
\usepackage{color}
\definecolor{darkblue}{rgb}{0.,0.,0.4}
\definecolor{darkred}{rgb}{0.5,0.,0.}
\usepackage[pdftex,colorlinks=true,linkcolor=darkblue,citecolor=darkred,urlcolor=blue]{hyperref}
\usepackage{xcolor}

\newcommand{\longcomment}[1]{}

\begin{document}
% \bibliographystyle{apa}
\fancypagestyle{nofooter}{\fancyfoot{}}
\thispagestyle{nofooter}  %  https://tex.stackexchange.com/questions/136384/remove-footer-without-removing-the-header
\setlength{\headheight}{30.0pt}
\fancyhead[C]{\includegraphics[height=0.5in]{stfx_text.png}}
\fancyhead[L]{\includegraphics[height=0.5in]{crest.png}}
\fancyhead[R]{
    \sffamily
    \scriptsize
    4130 University Avenue \\
    B2G 2W5 \\
    Antigonish, Nova Scotia \\
    Canada
}
\normalsize

\title{Cover Letter}
% \begin{center}
% \includegraphics[width=4in]{stfx_logo.png}
% \end{center}

\longcomment{
Specific comments:

% 1. Lines 42-45: I think it's unlikely that most readers of SPIE will be
% familiar with RMT and so it would be a good idea to have a bit more of an
% introduction - what's a random matrix?, what does RMT try to do (e.g.,
% characterize systems that can be described using matrices, statistically,
% rather than deterministically.) Later they talk about what RMT has been applied
% to, but it would be good to orient the reader to the general idea.

%%%%%% 2. Line 44: "independently" should be "independent"

%%%%%%% 3. Line 56: This is a pretty broad statement, because it would seem that it's
%%%%%%% an open question whether all complex systems would lend themselves to RMT-based
%%%%%%% analysis.

%%%%%% 4. Line 100: Do the authors really mean "eliminate" here? Do they instead mean
%%%%%% "mask", given that the information presumably doesn't disappear.

%%%%%% 5. Line 121: "functional connectivity". Do the authors mean "functional
%%%%%% connectivity matrix"?

%%%%%% 6. Line 138: This sentence is a bit garbled. Do the authors mean "often poorly
%%%%%% documented or even missing..."?

%%%%%% 7. Lines 143-145: I don't think these absences raise doubts about these two
%%%%%% things in general, but rather that one has to carefully document the workflows
%%%%%% and make the data and code available to the public.

%%%%%% 8. Line 156: Do the authors mean "allow a reasonable classification of each
%%%%%% scan without resorting to other information". Unclear what they mean by
%%%%%% "entirety".

%%%%%% 9. Table 1: Volumes and n_scans are not defined sufficiently. Are "Volumes" the
%%%%%% number of volumes per run? Is n_scans to total number of runs? Is there one run
%%%%%% per participant? The participant info is in Table 2, but Table 1 should be
%%%%%% independently clear.

%%%%%% 10. Line 180: "learning vs. rest" should be in a different font to be
%%%%%% consistent with the other dataset descriptions.

%%%%%% 11. Line 208: Are the "fractional anisotropy" values mentioned here the input
%%%%%% variables for BET which govern how aggressive BET is and the gradient
%%%%%% parameter?

%%%%%% 12. Section 3.2: While I understand that the authors are trying to make a
%%%%%% general statement that will apply to all features, this section is unclear
%%%%%% because the reader has no idea at this point what features are. Perhaps this
%%%%%% could be expanded.

%%%%%% 13. Line 223: How about "computing time" and "storage space", assuming that is
%%%%%% what is meant.

%%%%%% 14. Line 267: It's not clear that this section should be here, maybe at the end
%%%%%% of the methods section?

%%%%%% 15. Line 292: This is a minor point, but sentences shouldn't start with an
%%%%%% abbreviation (E.g.) better "For example,". This also occurs in line 387.

%%%%%% 16. Line 311: I see what they mean by "voxel time series" because of lines
%%%%%% 317-318, so it might be clearer to move those lines to directly after line 311,
%%%%%% so it is clear without having to read further.

17. The authors do a nice job of presenting the relevant results, but the
conclusions were rather general, being along the lines of "investigate the
sensitivity of choices made in the processing and analysis". That's an
important result, given previous work and shows that RMT might not "work" with
different data sets and processing workflows. The question is: what's the
takeaway: that the method is not one in which you can have a default set of
parameters that will generally give good results (like say, the recon-all
pipeline for FreeSurfer)? If RMT so sensitive to preprocessing and choice of
parameters, is this a useful method to be applied widely on fMRI data? It would
be good for this to be discussed and perhaps some recommendations given.
}

\noindent
\today, \\
Dear Dr. Gimi, \\

Thank you for your careful consideration of our manuscript
\emph{Random Matrix Theory Tools for the Predictive Analysis of Functional
Magnetic Resonance Imaging Examinations} submitted for publication in SPIE
Journal of Medical Imaging. We especially thank our anonymous reviewer,
who and offered especially constructive and specific comments on this manuscript.

The updated manuscript indicates changes with coloration, as suggested by
SPIE:JMI. Minor wording or other uncontroversial changes are identified in
\textcolor{orange}{orange} in the updated .pdf manuscript (or with
\verb|\textcolor{orange}{text}| in the \texttt{.tex} file). More significant
changes are identified in \textcolor{cyan}{cyan} (\verb|\textcolor{cyan}{text}|).

\begin{itemize}
\item Reviewer comments 2, 4, 5, 6, 10, 11, 13, 14, 15, 16 noted easily-correctable
and uncontroversial issues relating to phrasing, clarity or ordering, and we
have made changes in line with those comments on lines 55, 115, 136, 153,
196, 224, 247, 381-387, 413 and 309, and 329-331, respectively.

\item Comment 3 interprets our statement as making a broad claim, but the statement
was in fact intended to qualify or limit the scope of applicability of RMT.
This has been made clearer (lines 69-71). Likewise, we don't think there is
any disagreement between our intended meaning and the sentiment in comment 7,
and have revised line 159 to make it clear that our doubts are confined to
\textit{past} studies.

\item Reviewer comments 8 and 9 noted some issues with vagueness involving
descriptions of the fMRI scans and runs in the text and tables. We hope these
descriptions are now unambiguous (lines 169-174 and Table 1 caption).

\item In response to comment 12, while we did, and still do take as given that our
reviewer and a reader of SPIE:JMI knows what features and feature extraction
are, we can see how at this point in the paper, it may not be clear
\textit{why} we are extracting features. We have added a short paragraph (lines
234-241) that we hope resolves this, and makes it clear we are comparing the
predictive utility of different features / feature extraction methods (and that
``feature'' has the usual meaning in this context).
\end{itemize}


In response to our reviewer's first comment, we have expanded the introduction
to include both the definition of a random matrix, and some concrete examples of
random matrix ensembles or classes, and also a brief explanation of how RMT
theory is used in empirical contexts. We hope that this covers the basics both
in a way that presumes neither too little nor too much from readers (a very
difficult balance to strike with something like RMT, unfortunately).


Our reviewer's final comment suggested (by our reading) to give our conclusion
a bit more punch. The point is well-taken, and we have reworked our conclusion
to more strongly imply a need for caution and/or carefulness in the
presentation and interpretation of RMT-fMRI-related findings.


Please address all correspondence concerning this manuscript to \href{jlevman@stfx.ca}{jlevman@stfx.ca}.




\bigskip

\noindent
Sincerely, \\

\small
\noindent
\textbf{Corresponding Author}\\
Jacob Levman, PhD (jlevman@stfx.ca) \\
\footnotesize
\hspace*{0.25cm}St. Francis Xavier University, Department of Computer Science  \\
\hspace*{0.25cm}Martinos Center for Biomedical Imaging, Massachusetts General Hospital \\
\hspace*{0.25cm}Harvard Medical School, Department of Radiology (jlevman@mgh.harvard.edu) \\
\hspace*{0.25cm}Nova Scotia Health Authority, Research Affiliate (jacob.levman@nshealth.ca) \\
\normalsize

\small
\noindent
\textbf{Additional Contacts} [should the corresponding author be unavailable]\\
Derek Berger (dberger@stfx.ca)\\
\footnotesize
\hspace*{0.25cm} St. Francis Xavier University, Department of Computer Science \\
\small
Gurpreet S. Matharoo (gmatharo@stfx.ca) \\
\footnotesize
\hspace*{0.25cm} St. Francis Xavier University, Department of Physics \\
\hspace*{0.25cm} St. Francis Xavier University, ACENET
\normalsize



% \affil[b]{St. Francis Xavier University, ACENET, 4130 University Avenue, Antigonish, Nova Scotia, Canada, B2G 2W5}

% \affil[c]{St. Francis Xavier University, Department of Physics, 4130 University Avenue, Antigonish, Nova Scotia, Canada, B2G 2W5}

% \affil[d]{Martinos Center for Biomedical Imaging, 149 Thirteenth Street, Suite 2301, Charlestown, Massachusetts, United States, 02129}
% \affil[e]{Harvard Medical School, Department of Radiology, 25 Shattuck Street Boston, Massachusetts 02115, United States, 02129}
% \affil[f]{Nova Scotia Health Authority, Research Affiliate, Nova Scotia, Canada}

% \printbibliography % biblatex
% \renewcommand{\bibsection}{}  % https://tex.stackexchange.com/questions/132646/how-to-remove-the-references-title
% \smallskip
% \bibliographystyle{mystyle}
% \bibliography{cover}  % natbib

\footnotesize
\begin{center}
\noindent\rule{6cm}{0.4pt}
\end{center}

\AtNextBibliography{\footnotesize}
\printbibliography[heading=none]

\end{document}