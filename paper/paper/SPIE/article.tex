\documentclass[12pt]{spieman}  % 12pt font required by SPIE;
%\documentclass[a4paper,12pt]{spieman}  % use this instead for A4 paper
\usepackage{amsmath,amsfonts,amssymb}
\usepackage{graphicx}
\usepackage{setspace}
\usepackage{tocloft}
\usepackage{doi}

\title{Random Matrix Theory Tools for the Analysis of Functional Magnetic Resonance Imaging Examinations}

\author[a]{Derek Berger}
\author[a,*]{Jacob Levman}
\author[b]{Gurpreet M. Matharoo}
\affil[a]{St. Francis Xavier University, Department of Computer Science, 4130 University Avenue, Antigonish, Canada, B2G 2W5}
\affil[b]{St. Francis Xavier University, ACENET, 4130 University Avenue, Antigonish, Canada, B2G 2W5}

% \author[a]{First Author}
% \author[a]{Second Author}
% \author[b]{Third Author}
% \author[a,b,*]{Fourth Author}
% \affil[a]{University Name, Faculty Group, Department, Street Address, City, Country, Postal Code}
% \affil[b]{Company Name, Street Address, City, Country, Postal Code}

\renewcommand{\cftdotsep}{\cftnodots}
\cftpagenumbersoff{figure}
\cftpagenumbersoff{table}
\begin{document}
\maketitle

% \begin{abstract}
% This document shows the required format and appearance of a manuscript prepared for SPIE journals.
% It is prepared using LaTeX2e with the class file \texttt{spieman.cls}. Please note that the
% following journals require the use of structured abstracts in manuscript submissions:
% \textit{Neurophotonics}, the \textit{Journal of Biomedical Optics}, and the \textit{Journal of
% Medical Imaging}. Structured abstracts are encouraged for the \textit{Journal of
% Micro/Nanolithography, MEMS, and MOEMS}. Guidelines are available on the journal website. Whether
% structured or single-paragraph, the abstract should be a summary of the paper and not an
% introduction. Because the abstract may be used in abstracting and indexing databases, it should be
% self-contained (i.e., no numerical references) and substantive in nature, presenting concisely the
% objectives, methodology used, results obtained, and their significance. A list of up to six
% keywords should immediately follow.
% \end{abstract}

% See https://www.spiedigitallibrary.org/journals/journal-of-medical-imaging/volume-7/issue-01/010101/Integrating-structured-abstracts-in-the-Journal-of-Medical-Imaging/10.1117/1.JMI.7.1.010101.full
% for info on the structured abstract
%
% Purpose: This section presents the significance and aims, stating the broad impact and the
% rationale or motivation of the work, including potentially some background.
%
% Approach: This section briefly describes the materials and methods, including the statistical
% analyses, used in the research.
%
% Results: This section provides a core summary of the findings from the work, including study
% numbers, quantitative analyses, or discoveries.
%
% Conclusions: This section summarizes and interprets the approach and findings from the work,
% stating also the importance or impact of the findings.

% Purpose:
% Data-intensive modeling could provide insight on the broad variability in outcomes in
% spine surgery. Previous studies were limited to analysis of demographic and clinical
% characteristics. We report an analytic framework called “SpineCloud” that incorporates
% quantitative features extracted from perioperative images to predict spine surgery outcome.

% Approach:
% A retrospective study was conducted in which patient demographics, imaging, and outcome
% data were collected. Image features were automatically computed from perioperative CT.
% Postoperative 3- and 12-month functional and pain outcomes were analyzed in terms of improvement
% relative to the preoperative state. A boosted decision tree classifier was trained to predict
% outcome using demographic and image features as predictor variables. Predictions were computed
% based on SpineCloud and conventional demographic models, and features associated with poor outcome
% were identified from weighting terms evident in the boosted tree.

% Results
% Neither approach was predictive of 3- or 12-month outcomes based on preoperative data
% alone in the current, preliminary study. However, SpineCloud predictions incorporating image
% features obtained during and immediately following surgery (i.e., intraoperative and immediate
% postoperative images) exhibited significant improvement in area under the receiver operating
% characteristic (AUC): AUC=0.72 (CI95=0.59 to 0.83) at 3 months and AUC=0.69 (CI95=0.55
% to 0.82) at 12 months.

% Conclusions Predictive modeling of lumbar spine surgery outcomes was improved by incorporation of
% image-based features compared to analysis based on conventional demographic data. The SpineCloud
% framework could improve understanding of factors underlying outcome variability and warrants
% further investigation and validation in a larger patient cohort.
\begin{abstract}

\section*{Purpose}
Random matrix theory (RMT) is an increasingly useful tool for understanding large, complex systems.
Prior studies have examined functional Magnetic Resonance Imaging (fMRI) scans using tools from
RMT, with some success. However, RMT computations are complex, and sensitive to numerous analytic
choices, and previous investigations have been mostly explanatory / descriptive. We systematically
investigate the usefulness of RMT in a variety of fMRI datasets in a predictive framework.

\section*{Approach}
We develop open-source software to efficiently compute RMT features from fMRI images, and examine
the cross-validated predictive potential of eigenvalue and RMT-based features (``eigenfeatures'') used
with typical classificatory machine-learning algorithms. We systematically vary pre-processing extent,
normalization procedure, RMT unfolding procedure, and classifiers to examine the utility of these
eigenfeatures across various datasets.

\section*{Results}
We find eiegenfeatures to be broadly predictive in a wide variety of contexts, and that RMT-specific
features like the spectral rigidity and level number variance have considerable potential in
feature-engineering for fMRI. On median, we find ... However, we also find the utility of the
extracted eigenfeatures to depend strongly and complexly on various analytic decisions.

Also

\section*{Conclusions}
Consider using RMT.

% This document shows the required format and appearance of a manuscript prepared for SPIE journals.
% It is prepared using LaTeX2e with the class file \texttt{spieman.cls}. Please note that the
% following journals require the use of structured abstracts in manuscript submissions:
% \textit{Neurophotonics}, the \textit{Journal of Biomedical Optics}, and the \textit{Journal of
% Medical Imaging}. Structured abstracts are encouraged for the \textit{Journal of
% Micro/Nanolithography, MEMS, and MOEMS}. Guidelines are available on the journal website. Whether
% structured or single-paragraph, the abstract should be a summary of the paper and not an
% introduction. Because the abstract may be used in abstracting and indexing databases, it should be
% self-contained (i.e., no numerical references) and substantive in nature, presenting concisely the
% objectives, methodology used, results obtained, and their significance. A list of up to six keywords
% should immediately follow.
\end{abstract}

% Include a list of up to six keywords after the abstract
\keywords{random matrix, spectral rigidity, level number variance, fMRI, classification, machine-learning}

% Include email contact information for corresponding author
{\noindent \footnotesize\textbf{*}Jacob Levman,  \linkable{jlevman@stfx.ca} }

\begin{spacing}{2}   % use double spacing for rest of manuscript

\section{Introduction}
\label{sect:intro}  % \label{} allows reference to this section
% This document shows the format and appearance of a manuscript prepared for submission to an SPIE
% journal. Note that this template is only intended to be used as a guideline for author convenience.
% It is designed for optimum clarity and ease of reading for editors and reviewers, but the template
% does not reflect the final page layout of a published journal paper. Accepted papers are
% professionally typeset in XML according to the layout and design of the journal.

\textbf{NOTE}: Forget academia for a moment, and write like its a blog. Then go back and add
the citations and adjust the tone.

Random Matrix Theory (RMT) has obvious potential. It has helped to understand the generalization
behaviour of neural networks [add citation], can be used to guide any principal components analysis
reduction [cite: Google search: "random matrix" "principal components"], has been used to help
understand phenomena as diverse as [include the usual citations to buses, traffic, physics, etc],
and can provide insights wherever there is a system of components interacting with some
independence [cite Cambridge RMT handbook, others].

At its most rudimentary, RMT simply describes the statistically-expected behaviour of the
eigenvalues of some classes—or \textit{ensembles}—of random matrices under certain assumptions or
conditions. A random matrix is, generally, a matrix in which the entries can be considered to be
independently and identically distributed (\textit{iid}). For example, one of the simplest ensembles [cite handbook] is the Gaussian
Unitary Ensemble (GUE), wherein [desc of ensemble properties] and which appears to be useful in
describing [real-world examples].

This means that if we examine a system and find that the distribution of eigenvalues of that system
resemble some RMT predicted spectrum, we potentially gain insight into the nature of that system.
If, in addition, we find a large deviance from the spectrum predicted by RMT, we find contrapositive
evidence that the system is \textit{not} like those described by RMT. Roughly, complex systems
well-described by RMT are highly random and independent, whereas systems that are not so
well-described are either highly dependent, non-random, or both.

In this sense, RMT might have real \textit{explanatory} and \textit{statistical} value: the more we
see that a system resembles an RMT system, the less we can distinguish it from one of random,
non-interacting components, and the less the system resembles an RMT system, the more
fundamentally complex and interactive the interactions.

For a system as complex as the human brain, RMT is thus tantalizing. It would be quite something
if we could extract some matrix reduction of brain activity, and if the RMT description of this
matrix either aligned nicely with RMT theory, or otherwise was broadly predictively useful.

[Transition to fMRI now...]


In functional magnetic resonance imaging (fMRI), changes in the blood-oxygenation-level-dependent
(BOLD) signals are related to neural activity. It is common to investigate statistical relationships
between the BOLD signals through \textbf{functional connectivity} analyses, where correlations between
collections of these signals are examined to infer connections between different voxels or regions
of interest (ROIs) within the brain.

Whether in the presence of experimental stimuli, or the relative absence, as in resting state fMRI (rs-fMRI),
complex patterns in functional connectivity networks are ubiquitous
\cite{bucknerBrainDefaultNetwork2008,foxCoverHumanBrain2005,gonzalez-castilloTaskbasedDynamicFunctional2018,hermundstadStructuralFoundationsRestingstate2013}.
This complexity suggests fMRI is a candidate to be studied using \textbf{Random Matrix Theory} (RMT), a set
of mathematical tools originally developed some 50 years ago to solve complex problems in nuclear
physics \cite{guhrRandommatrixTheoriesQuantum1998a,mehtaRandomMatrices2004}

In various
physical systems where nuclei can have different possible states, precisely modeling the entirety of
the interactions between nuclei is often intractable. For such large, complex systems, the
eigenvalues (or spectra) efficiently summarize the totality of the interactions between components
of the system, and RMT describes the expected behaviour of such eigenvalues. By analysing the
spectrum statistically, complex systems can be investigated by comparison to the universal
properties predicted by RMT.

\subsection{RMT and Neurobiological Signals}

Recently, RMT has been applied to study brain functioning. The earliest study
demonstrated that spectra of the correlations between electroencephalographic (EEG) signals closely
resemble those of the Gaussian Orthogonal Ensemble (GOE) \cite{sebaRandomMatrixAnalysis2003}.
Somewhat more recently, RMT was used to evaluate the quality of whole brain features extracted from
fMRI data \cite{voultsidouFeatureEvaluationFMRI2007,verganiRestingStateFMRI2019}. RMT has also
been used in diffusion MRI to aid in the selection of the number of components to employ in
principal-component reduction analysis and denoising
\cite{veraartDenoisingDiffusionMRI2016,verganiRestingStateFMRI2019,ulfarssonDimensionEstimationNoisy2008}.

Finally, RMT has been used in ROI-based fMRI functional-connectivity studies to investigate
differences between rest and task states \cite{wangSpectralPropertiesTemporal2015}, between
subjects with and without ADHD \cite{wangRandomMatrixTheory2016}, and between pain and non-pain
states \cite{matharooSpontaneousBackpainAlters2020}. Across these three studies, the spectra of
resting or low-attention states exhibited properties closest to the GOE. These findings suggest that
certain aspects of psychological processes might be characterized, in part, by features computed
from the eigenvalues of fMRI correlation matrices, and that these features might vary in an
interpretable manner across psychological processes. If this is the case, RMT could aid
in understanding the functioning of the human brain.


\subsection{Motivating Eigenfeatures}

The basic insight of RMT is thus that the \textit{eigenvalues alone} may provide interesting
information about highly complex Neurobiological systems. However, in practice, the eigenvalues computed from
empirically observed data are noisy. RMT is statistical in nature, describing only the
\textit{expected} behaviour of the system spectra. To this end, a number of summary statistics or
``spectral observables'' \cite{mehtaRandomMatrices2004} can be computed from empirically observed
spectra, with these spectral observables sometimes being better suited for further analysis or
comparison to theory. Two such summary statistics that have been somewhat popular are the
\textit{spectral rigidity} and \textit{level number variance} (``rigidity'' and ``level variance''
for short; details in Section \ref{sect:methods}).

However, from a predictive standpoint, these summary statistics may eliminate predictively-useful
information. If the basic RMT insight is that the eigenvalues are useful features to understand a
system, then those eigenvalues alone (or other simple transformations of them) ought also to be
predictively useful. We examine a number of such \textit{eigenfeatures} in this study.

In addition, the functional connectivity - the matrix of correlations of various collections of
voxels of an fMRI scan - is a priori a useful representation of the fMRI data. However, the full
correlation matrix between all voxels is itself often computationally infeasible to work with, and
thus is reduced in various ways prior to being used in analyses.

Typical reductions include the use of a ``seed'' voxel or collection of voxels (i.e. ROI): the mean
signal over that ROI is correlated with all other N ROIs of interest to generate a reduced functional
connectivity matrix (or image) with one correlation value at each non-seed ROI. This sort of reduction reduces the functional connectivity to N values. This approach strongly
biases the analysis to the seed ROI.

Likewise, one can work with ROI mean signals, and work with the \(N \times N\) matrix of
these ROI correlations. This analysis [maybe cite Poldrack book for this?] is, \textit{a priori},
sensitive to the choice of ROIs, and, in the case of anatomical atlases / parcellations, may also
lack theoretical justification.

A major disadvantage of both the above approaches, besides requiring parcellation into ROIs, is
that most such parcellation methods are atlas-based, and thus require image registration, and,
general, an entire preprocessing pipeline to ensure that registration is successful.

RMT suggests a potentially useful reduction in the form of the eigenvalues of the voxelwise
functional connectivity. An \(N \times t\) matrix \(\bf{A}\) of \(N\) time series of length \(t\),
and where \(t \ll N\) will have a symmetric correlation (or covariance) matrix with \(t - 1\)
non-zero, positive real-valued eigenvalues \(\Lambda\). These eigenvalues can be computed highly
efficiently via transposition of the voxelwise correlation or covariance matrix.


\subsection{Limitations of Previous Work}

However, these previous studies took largely descriptive / explanatory approaches, noting only
differences in RMT across subgroups and/or conditions, in many cases, with no formal statistical
model nor formal testing of the significance and robustness of these differences. No code was provided to
allow the reproduction of the results, and the extraction of the spectral rigidity and level
variance is non-trivial, and computationally demanding. Computation of these spectral observables
requires an \textit{unfolding}
procedure\cite{guhrRandommatrixTheoriesQuantum1998a,mehtaRandomMatrices2004}, which is,
unfortunately, a poorly-documented fitting procedure that requires making a number of
difficult-to-justify decisions, and these decisions can dramatically impact RMT conclusions
\cite{abul-magdUnfoldingSpectrumChaotic2014,abueleninSpectralUnfoldingChaotic2018,fossionRandommatrixSpectraTime2013,abueleninEffectUnfoldingSpectral2012,moralesImprovedUnfoldingDetrending2011}.
On top of the flexibility in the implementation and application of RMT to empirical data, there
is also the flexibility introduced by the choices in the complex preprocessing pipelines of fMRI\cite{parkerBenefitSliceTiming2019}.

In addition these issues stemming from ``research degrees of freedom'' (cite Gelman), there is also
a basic baseline comparison issue: RMT is mathematically non-trivial, the extraction of RMT
spectral statistics is computationally demanding, and eigenvalues, while easy to understand
mathematically, are difficult to interpret relative to simpler potential reductions of fMRI data.

What is missing is a rigorous, systematic, reproducible investigation of the *predictice* value of
RMT metrics across a wide variety of data, RMT analytic choices, choice of eigenfeature, and choice
of preprocessing, and for these RMT eigenfeatures to be compared to simpler alternative baselines.
The current study aims to do just this.


\section{Datasets}

\subsection{Overview}

In order not to inundate readers with dataset details, and because this is an exploratory
investigation of RMT which is not committed to any specific theory, we refer the reader
to the original publications and/or data releases when greater detail is needed, and
highlight only the most basic aspects of each here.

We selected fMRI data publicly available on the \href{https://openneuro.org/}{OpenNeuro} platform
(36). Selection criteria was somewhat subjective, but we required (1), that all scans have the same
spatial and temporal resolutions, (2), the the dataset comprise 10+ human subjects, and (3), that it
was reasonable to classify at the level of an entire scan, i.e. neither task nor event timings were
needed to determine class membership. This yielded seven datasets comprising XX total subjects, and
over 5000 total fMRI volumes.

Dataset scan parameters and acquisition details are summarized in Table \ref{table:1}, and sample
and subgroup sizes are summarized in Table \ref{table:2}. For all datasets, we use the raw
(unpreprocessed) scans, and perform our own preprocessing (detailed later). Given the complexity of
these datasets, we highlight below only those aspects relevant to the current paper.

\subsubsection{Aging}
This dataset\cite{wahlheimIntrinsicFunctionalConnectivity2021} included 34 subjects with a mean age
of 22 years (range: 18-32 years), and 28 subjects with a mean age of 70 years (range: 61-80 years).

% The final sample included 34 younger adults (1832 years old, Mage = 22.21, SD = 3.65; 20 female)
% and 28 older adults (61-80 years old, Mage = 69.82, SD = 5.64; 20 female). Table 1 shows that
% younger and older adults had comparable education and working memory capacity, the latter measured
% by forward and backward digit span (48). Older adults had higher vocabulary scores (49) and slower
% processing speed (50) than younger adults.
% Resting-state data comprised 300 measurements collected over one 10-minute run. This scan duration
% produces reliable functional connectivity estimates within and across participants (57).
% Participants were instructed to remain still and awake with their eyes open. No stimuli appeared
% during this scan, and the display was black. Functional scans were collected using an echo-planar
% image sequence sensitive to blood-oxygen-level-dependent (BOLD) contrast (T2*; 32 slices with 4.0
% mm thickness and no skip, TE = 30 ms, TR = 2000 ms, flip angle = 70, FOV = 220 mm, matrix size =
% 74 × 74 × 32 voxels, A/P phase encoding direction). Slices were collected in a descending order
% that covered the entire cortex and partial cerebellum. At the beginning of the resting-state scan,
% the scanner acquired and discarded two dummy scans.

\subsubsection{Bilinguality}

This dataset\cite{ds001747:1.0.0, goldcarrieelizabethExploringRestingState2018} examined English and
Spanish-speaking monolinguals and multilinguals during a prolonged resting state. The original study
grouped participants into three classes: early vs late bilinguals vs monolingual controls, and found
connectivity differences only between the early and late
bilinguals\cite{goldcarrieelizabethExploringRestingState2018}. We compared for our two classes just
"bilingual" vs "monolingual". The data is most notable for the short TR (875ms) combined with long
acquisition time (12 minutes).

% Resting state fMRI scans were performed using a T2*-weighted EPI sequence with the following
% parameters: TR=875ms; TE=43.6ms; slice thickness=1.8mm; acquisition matrix=100 × 100; flip
% angle=55°; number of slices=72; field of view=180 × 180mm; voxel size=1.8 × 1.8 × 1.8mm;
% multi-band factor=8; volumes per run=823 (total scan time=12min). EPI scans were oriented
% parallel to the long axis of the hippocampus.

\subsubsection{Learning}

This dataset\cite{schapiroHumanHippocampalReplay2018, schapiroHumanHippocampalReplay2020}. In this study, both rs-fMRI and task fMRI scans
were available for all subjects, allowing for within-subject comparisons. The task is complex and difficult to summarize here adequately, but involved
multiple phases where subjects were presented with various related abstract images, and then later tested for their memory of certain aspects of the learned images. For our study, we simply note that these task scans ought to be broadly more attentionally  and cognitively demanding than the resting state, and instead simply try to classify task vs. test scans using eigenfeatures.



\subsubsection{Dataset: PSYCH}
The PSYCH Dataset  is detailed in \cite{gorgolewskiHighResolution7Tesla2015}. Subjects completed a
battery of psychological measurements, and various resting-state fMRI (rs-fMRI) whole-brain 7-Tesla
scans were obtained. For the purpose of keeping with the ``attention'' theme of the analyses in this
paper, three pairings of subgroups were constructed via z-score splits of the various sum-scores of
scale items involving concepts closely related to attention.

\emph{Trait Attention}.
The PANAS-X \cite{watsonPANASXManualPositive1994} is a 9-point Likert-type self-report measure that
queries the extent to which subjects agree particular affective terms have applied to them over a
recent time period (e.g. a few weeks). An ad hoc (i.e. face-valid) scale was constructed from the
scale items: ``attentive'', ``alert'', ``interested'', ``concentrating'', and reverse-scored items
``sluggish'', ``tired'', ``sleepy'', ``drowsy''. Subjects were then placed in the high / low ``trait
attention'' category based on a median split of the average z-score across items.

\emph{Task Attention}.
The Conjunctive Continuous Performance Task \cite{shalevConjunctiveContinuousPerformance2011} (CCPT)
 is a behavioural measure of sustained attention
where subjects must react only if they see a red square among other different coloured figures. For
this paper, a sum score was constructed by subtracting the number of false positive and false
negative responses from the number of success responses, and subjects with a z-score greater / less
than zero on this scale were placed in high / low ``task attention'' category.

\emph{Vigilance}.
The Mini New York cognition questionnaire assesses various elements of the subject's cognitive
state, e.g. ``my thoughts involved other people'', ``I thought about something positive'' (translated
from German; \cite{gorgolewskiCorrespondenceIndividualDifferences2014}. One question assesses
alertness (``I was fully awake'' / ``war ich vollkommen wach'') and was used to create high / low
``vigilance'' conditions in the same manner as the previous two splits.

\subsubsection{Dataset 2}
Dataset 2 (referred to in figures and tables with the label ``OSTEO'') is detailed in
\cite{tetreaultBrainConnectivityPredicts2016}. Subjects included healthy controls (``nopain''
condition) and individuals with knee osteoarthritis (``pain'' condition) treated for 2 weeks with
either placebo (``pain'' condition) or duloxetine (``duloxetine'' condition). An additional grouping was
also created with both the placebo and duloxetine data (``allpain'' condition), allowing for six
subgroup comparisons. All subjects were imaged with whole-brain rs-fMRI. Scan parameters are
documented in Table \ref{table:1}. The use of rs-fMRI and RMT for examining pain makes this dataset
complementary to \cite{matharooSpontaneousBackpainAlters2020}.


\subsubsection{Dataset 4}
Dataset 4 (referred to in figures and tables with the label ``PARK'' or ``PARKINSONS'') is documented and
described in detail in \cite{madhyasthaDynamicConnectivityRest2015}. Scan parameters are documented
in Table \ref{table:1}. Subjects with non-demented Parkinson's Disease (PD) and healthy controls performed a
number of repetitions of the Attention Network Task
(ANT; \cite{fanActivationAttentionalNetworks2005} ). Subgroup details are available in Table \ref{table:1}. As the ANT
task demands high attention, it is not as straightforward to create ``high'' or ``low'' attention
groups. Instead, since attention presumably is high and varies little in this dataset, and subgroups
instead differ by virtue of the presence or absence of PD, any results here serve as a useful
comparison point for the results from studies that should clearly involve attentional differences.
This dataset also included extensively preprocessed scans, which were included in analyses to
examine the effects of preprocessing on the extracted RMT features.

\subsubsection{Dataset 5}

\begin{table}[h!]
\caption{
    ID = Identifier for paper. TR = Time of Repetition (seconds).
    Time = total duration (minutes) of each scan. Dimensions listed as M x N x P,
    indicate P horizontal / axial slices each with dimensions M x N.
}
\label{table:1}
\small
\centering
\begin{tabular}{ c c c c c c c }
\hline
\textbf{ID}    & \textbf{Dimensions}  & \textbf{Voxel size (mm)} & \textbf{TR} & \textbf{Volumes} & \textbf{n\_scans} \\
\hline
AGING     & 74 x 74 x 32   & 3.0 x 3.0 x 4.0 & 2.0  & 300 & 62  \\
BILING    & 100 x 100 x 72 & 1.8 x 1.8 x 1.8 & 0.88 & 823 & 90  \\
DEPRESS   & 112 x 112 x 25 & 2.0 x 2.0 x 5.0 & 2.5  & 100 & 72  \\
LEARN     & 64 x 64 x 36   & 3.0 x 3.0 x 3.0 & 2.0  & 195 & 432 \\
OSTEO     & 64 x 64 x 36   & 3.4 x 3.4 x 3.0 & 2.5  & 300 & 74  \\
PARK      & 80 x 80 x 43   & 3.0 x 3.0 x 3.0 & 2.4  & 149 & 552 \\
PSYCH     & 128 x 128 x 70 & 1.5 x 1.5 x 1.5 & 3.0  & 300 & 90  \\
\hline
\end{tabular}


\end{table}

\begin{table}[h!]
\caption{
    ID = Identifier for paper. Subgroup = name of subgroup. Subjects = Number of subjects.
    Task = fMRI task. ANT = Attention Network Task \cite{fanActivationAttentionalNetworks2005}
}
\label{table:2}
\small
\centering
\begin{tabular}{ l l c c c }
\hline
\textbf{ID}  & \textbf{Subgroup}  & \textbf{Subjects}  & \textbf{Task} & \textbf{Scans per Subject} \\
\hline
AGING   &  older             & 28      & resting-state         & 1 \\
        &  younger           & 34      & resting-state         & 1 \\
\hline
BILING  &  bilingual          & 59      & resting-state        & 1 \\
        &  monolingual        & 33      & resting-state        & 1 \\
\hline
DEPRESS &  depression         & 51      & resting-state        & 1 \\
        &  control            & 21      & resting-state        & 1 \\
\hline
LEARN   &  task              & 24      & learn image sketches  & 16 \\
        &  rest              & 24      & resting-state         & 2 \\
\hline
OSTEO   &  duloxetine        & 19      & taking duloxetine     & 1 \\
        &  pain              & 37      & placebo               & 1 \\
        &  nopain            & 20      & control               & 1 \\
\hline
PARK    &  Parkinson's       & 25      & ANT                   & 12 \\
        &  control           & 21      & ANT                   & 12 \\
\hline
PSYCH   &  vigilant           & 11     & resting-state        & 3 \\
        &  nonigilant         & 11     & resting-state        & 3 \\
        &  trait-attentive     & 11    & resting-state        & 3 \\
        &  trait-non-attentive & 11    & resting-state        & 3 \\
        &  task-attentive     & 11     & resting-state        & 3 \\
        &  task-nonattentive  & 11     & resting-state        & 3 \\
\hline
\end{tabular}
\end{table}

\section{Methods}
\label{sect:methods}

Given empirically observed eigenvalues \(\Lambda\), the spectral rigidity, \(\Delta_3(L)\), is
calculated for any positive real value \(L < \max(\Lambda)\) as:
\[
\Delta_3(L) = \left \langle \min_{A,B} \frac{1}{L} \int_c^{c+L} \left(  \eta(\lambda) -A \lambda - B \right)^2 \right \rangle_c
\]
where \(\eta(\lambda)\) is the number of unfolded eigenvalues less than or equal to \(\lambda\),
\(\langle \cdot \rangle_c\) denotes the average with respect to all starting points \(c\), and where
\(A\) and \(B\) denote the slope and intercept, respectively, of the least squares fit of a straight
line to \(\eta(\lambda)\) on \([c, c+L]\) \cite{guhrRandommatrixTheoriesQuantum1998a}. The spectral
rigidty for a value \(L\) is thus the \textit{average nonlinearity}


The level
number variance, \(\Sigma^2(L)\), is closely related to the spectral rigidity
\cite{mehtaRandomMatrices2004}, and is calculated as:

\[
\Sigma^2(L) = \left\langle \eta^2(L, c) \right\rangle_c - \left\langle \eta(L, c) \right\rangle^2_c
\]

where \(\eta(L, c)\) is the number of unfolded eigenvalues in \([c, c+ L]\), and where \(c\), \(L\),
and \(\langle \cdot \rangle_c\) are as above \cite{guhrRandommatrixTheoriesQuantum1998a}. Both the
spectral rigidity and level number variance are also computed and investigated in this study.

\section{Results}
\label{sect:results}

First not non-predictable data, and then no longer discuss it in subsequent features.

\subsection{Normalization}
NO meaningful effect of normalization on AUROC distributions at level of gross feature, or feature
group, so can safely ignore this.
NO meaningful effect of normalization for any feature group by classifier either, so
normalization doesn't seem to matter here.
Maybe add table of correlations by groupings here

\subsection{Classifier}
When the data is predictable, timeseries features almost all have the mode / max <= 0.5 for all
classifiers and subgroups, whereas eigenfeatures (RMT, eigenvalue only, eigenvalue + RMT) more often
than not have mode greater than 0.5, and bulk of the distribution > 0.5.

KNN3 consistently has least favourable AUROC distributions across subgroups and features

\subsection{Page Setup and Fonts}

All text and figures, including footnotes, must fit inside a text area 6.5 in.\ wide by 9 in.\ high (16.51 by 22.86 cm). Manuscripts must be formatted for US letter paper, on which the margins should be 1 in.\ (2.54 cm) on the top, 1 in.\ on the bottom, and 1 in.\ on the left and right.

The Times New Roman font is used throughout the manuscript, in the sizes and styles shown in Table~\ref{tab:fonts}. If this font is not available, use a similar serif font. The manuscript should not contain headers or footers. Pages should be numbered.

\begin{table}[ht]
\caption{Fonts sizes and styles.}
\label{tab:fonts}
\begin{center}
\begin{tabular}{|l|l|} %% this creates two columns
%% |l|l| to left justify each column entry
%% |c|c| to center each column entry
%% use of \rule[]{}{} below opens up each row
\hline
\rule[-1ex]{0pt}{3.5ex}  Document entity & Brief description  \\
\hline\hline
\rule[-1ex]{0pt}{3.5ex}  Article title & 16 pt., bold, left justified  \\
\hline
\rule[-1ex]{0pt}{3.5ex}  Author names & 12 pt., bold, left justified   \\
\hline
\rule[-1ex]{0pt}{3.5ex}  Author affiliations & 10 pt., left justified   \\
\hline
\rule[-1ex]{0pt}{3.5ex}  Abstract & 10 pt.  \\
\hline
\rule[-1ex]{0pt}{3.5ex}  Keywords & 10 pt.  \\
\hline
\rule[-1ex]{0pt}{3.5ex}  Section heading & 12 pt., bold, left justified  \\
\hline
\rule[-1ex]{0pt}{3.5ex}  Subsection heading & 12 pt., italic, left justified  \\
\hline
\rule[-1ex]{0pt}{3.5ex}  Sub-subsection heading & 11 pt., italic, left justified  \\
\hline
\rule[-1ex]{0pt}{3.5ex}  Normal text & 12 pt. \\
\hline
\rule[-1ex]{0pt}{3.5ex}  Figure and table captions &  10 pt. \\
\hline
\end{tabular}
\end{center}
\end{table}

\section{Parts of Manuscript}

This section describes the normal structure of a manuscript and how each part should be handled. The appropriate vertical spacing between various parts of this document is achieved in LaTeX through the proper use of defined constructs, such as \verb|\section{}|.

\subsection{Title and Author Information}
\label{sect:title}
The article title appears left justified at the top of the first page. The title font is 16 pt., bold. The rules for capitalizing the title are the same as for sentences; only the first word, proper nouns, and acronyms should be capitalized. Do not begin titles with articles (for example, a, an, the) or prepositions (for example, on, by, etc.). The word ``novel'' should not appear in the title, as publication will imply novelty. Avoid the use of acronyms in the title, unless they are widely understood. Appendix A contains more about acronyms.

The list of authors immediately follows the title, 18 points below. The font is 12 pt., bold and the author names are left justified. The author affiliations and addresses follow the names, in 10-pt., normal font and left justified. For multiple affiliations, each affiliation should appear on a separate line. Superscript letters (a, b, c, etc.) should be used to associate multiple authors with their respective affiliations. The corresponding author should be identified with an asterisk, and that person's email address should be provided below the keywords.

\subsection{Abstract}
The abstract should be a summary of the paper and not an introduction. Because the abstract may be used in abstracting journals, it should be self-contained (i.e., no numerical references) and substantive in nature, presenting concisely the objectives, methodology used, results obtained, and their significance. Please note that the following journals require the use of structured abstracts in manuscript submissions: \textit{Neurophotonics}, the \textit{Journal of Biomedical Optics}, and the\textit{ Journal of Medical Imaging}. Structured abstracts are encouraged for the \textit{Journal of Micro/Nanolithography, MEMS, and MOEMS}. Helpful guidelines for structured abstracts are available on the website of the journal.

\subsection{Subject terms/Keywords}
Keywords are required. Please provide 3-6 keywords related to your paper.

\subsection{Body of Paper}
The body of the paper consists of numbered sections that present the main findings. These sections should be organized to best present the material.

To provide transition elements in your paper, it is important to refer back (or forward) to specific sections. Such references are made by indicating the section number, for example, ``In Sec.\ 2 we showed...'' or ``Section 2.1 contained a description...'' If the word Section, Reference, Equation, or Figure starts a sentence, it is spelled out. When occurring in the middle of a sentence, these words are abbreviated Sec., Ref., Eq., and Fig.

At the first occurrence of an acronym, spell it out followed by the acronym in parentheses, for example, charge-coupled diode (CCD).

\subsection{Footnotes}
Textual footnotes should be used rarely to present important documentary or explanatory material whose inclusion in the text would be distracting.\footnote{Example of a footnote.} Due to problems with HTML display, use of footnotes should generally be avoided. If absolutely necessary, the footnote mark must come at the end of a sentence. To insert a footnote, use the {\verb|\footnote{}|} command.

\subsection{Appendices}
Brief appendices may be included when necessary, such as derivations of equations, proofs of theorems, and details of algorithms. Equations and figures appearing in appendices should continue sequential numbering from earlier in the paper.

\subsection{Disclosures}
Conflicts of interest should be declared under a separate header, above Acknowledgments. If the authors have no competing interests to declare, then a statement should be included declaring no conflicts of interest. For assistance generating a disclosure statement, see the form available from  the International Committee of Medical Journal Editors website: \linkable{http://www.icmje.org/conflicts-of-interest/}

\subsection{Acknowledgments}
Acknowledgments and funding information should be added after the conclusion, and before references. Include grant numbers and the full name of the funding body. The acknowledgments section does not have a section number.

\subsection{Data, Materials, and Code Availability}
Availability of data, materials, and/or code used in the research results reported in the manuscript may be declared under the heading ``Data, Materials, and Code Availability,'' following the Acknowledgments section. As relevant, provide specific access information or restrictions for data, materials, and computer code (i.e., links to repository access addresses with guidance on commercial or public access).

\subsection{References}
The References section lists books, articles, and reports that are cited in the paper. This section does not have a section number. The references are numbered in the order in which they are cited. Examples of the format to be followed are given at the end of this document.

The reference list at the end of this document is created using BibTeX, which looks through the file {\ttfamily report.bib} for the entries cited in the LaTeX source file.  The format of the reference list is determined by the bibliography style file {\ttfamily spiejour.bst}, as specified in the \\ \verb|\bibliographystyle{spiejour}| command.  Alternatively, the references may be directly formatted in the LaTeX source file.

For books\citenum{Lamport94,Alred03,Goossens97} the listing includes the list of authors (initials plus last name), book title (in italics), page or chapter numbers, publisher, city, and year of publication.  Journal-article references \citenum{Metropolis53,Harris06} include the author list, title of the article (in quotes), journal name (in italics, properly abbreviated), volume number (in bold), inclusive page numbers or citation identifier, and year.  A reference to a proceedings paper or a chapter in an edited book\citenum{Gull89a} includes the author list, title of the article (in quotes), conference name (in italics), editors (if appropriate), volume title (in italics), volume number if applicable (in bold), inclusive page numbers, publisher, city, and year.  References to an article in the SPIE Proceedings may include the conference name, as shown in Ref.~\citenum{Hanson93c}.

The references are numbered in the order of their first citation. Citations to the references are made using superscripts, as demonstrated in the preceding paragraph. One may also directly refer to a reference within the text, for example, ``as shown in Ref.~\citenum{Metropolis53} ...''  Two or more references should be separated by a comma with no space between them. Multiple sequential references should be displayed with a dash between the first and last numbers \citenum{Alred03,Perelman97,Lamport94,Goossens97,Metropolis53}.

\subsubsection{Reference linking and DOIs}
A Digital Object Identifier (DOI) is a unique alphanumeric string assigned to a digital object, such as a journal article or a book chapter, that provides a persistent link to its location on the internet. The use of DOIs allows readers to easily access cited articles. Authors should include the DOI at the end of each reference in brackets if a DOI is available. See examples at the end of this manuscript. A free DOI lookup service is available from CrossRef at \\\linkable{http://www.crossref.org/freeTextQuery/}. The inclusion of DOIs will facilitate reference linking and is highly recommended.

In the present LaTeX template, the author needs to add the DOI reference by including it in a ``note'' in the bibliography file, as shown in the file {\verb+report.bib+}, for example, \\ {\verb+note = "[doi:10.1117/12.154577]"+}. The DOI may be used by the reader to locate that document with the link: {\verb+http://dx.doi.org10.1117/12.154577+}.

\subsection{Biographies}
A brief professional biography of approximately 75 words may be provided for each author, if available. Biographies should be placed at the end of the paper, after the references. Personal information such as hobbies or birthplace/birthdate should not be included. Author photographs are not published.

\section{Section Formatting}
\label{sect:sections}
In LaTeX, a new section is created with the \verb|\section{}| command, which automatically numbers the sections. Sections will be numbered sequentially, starting with the first section after the abstract, except for the acknowledgments and references. (Note that numbering of section headings is not required, but the numbering must be consistent if used.) All section headings should be left justified.

Main section headings are in 12-pt. bold font, left-justified and in title case, where important words are capitalized.

Paragraphs that immediately follow a section heading are leading paragraphs and should not be indented, according to standard publishing style. The same goes for leading paragraphs of subsections and sub-subsections. Subsequent paragraphs are standard paragraphs, with 0.2-in (5 mm) indentation. There is no additional space between paragraphs. In LaTeX, paragraphs are separated by blank lines in the source file. Indentation of the first line of a paragraph may be avoided by starting it with \verb|\noindent|.

\subsection{Subsection Headings}
All important words in a subsection (level 1) header are capitalized. Subsection numbers consist of the section number, followed by a period, and the subsection number within that section, without a period at the end. The heading is left justified and its font is 12 pt. italic.

\subsubsection{Sub-subsection headings}
The first word of a sub-subsection is capitalized. The rest of the text is not capitalized, except for proper names and acronyms (the latter should only be used if well known). The heading is left justified and its font is 11 pt. italic.

\section{Figures and Tables}

\subsection{Figures}

Figures are numbered in the order in which they are called out in the text. They should appear in the document in numerical order and as close as possible to their first reference in the text. It may be necessary to move figures or tables around to enhance readability. LaTeX will attempt to place figures at the top or bottom of a page in which they are first referenced.

Figures, along with their captions, should be separated from the main text by  0.2 in.\ or 5 mm and centered. Figure captions are centered below the figure or graph. Figure captions start with the abbreviation ``Fig'' in front of the figure number, followed by a period, and the text in 10-pt. font. See Fig.~\ref{fig:example} for an example.

\begin{figure}
\begin{center}
\begin{tabular}{c}
\includegraphics[height=5.5cm]{mcr3b.eps}
\end{tabular}
\end{center}
\caption
{ \label{fig:example}
Example of a figure caption. }
\end{figure}

Authors may wish to create figures consisting of two or more images, in which case, they should be neatly arranged in a rectangular array.  In no case, should the article's text be wrapped around a figure. Figure~\ref{fig:example2} shows two side-by-side images. When a figure contains more than one image, the author must submit them as a single image file. Further details about figure formatting can be found in the author guidelines for each specific SPIE journal: \\
\linkable {https://www.spiedigitallibrary.org/journals/journal-authors}.

\begin{figure}
\begin{center}
\begin{tabular}{c}
\includegraphics[height=5.5cm]{fig2.eps}  % fig2 includes two images
\\
(a) \hspace{5.1cm} (b)
\end{tabular}
\end{center}
\caption
{ \label{fig:example2}
Example of a figure containing multiple images: (a) sun and (b) blob. Figures containing multiple images must be submitted to SPIE as a single image file.}
\end{figure}

\subsection{Tables}
Tables are numbered in the order in which they are referenced. They should appear in the document in numerical order and as close as possible to their first reference in the text. It is preferable to have tables appear at the top or bottom of the page, if possible. Table captions are handled identically to those for figures, except that they appear above the table. See Table~\ref{tab:fonts} for an example.

\subsection{Multimedia}
Acceptable file formats, including MOV (.mov), MPEG (.mpg), and MP4 (.mp4), are playable using standard media players, such as VLC or Windows Media Player. The recommended maximum size for each multimedia file is 10-12 MB. Authors must insert a representative still image from the video file in the manuscript as a figure. The caption label will be linked by the publisher to the actual video file. The video may also be mentioned in an existing figure caption. Multimedia files are treated in the same manner as figures and they will be numbered sequentially with normal figures.  The video number, file type, and file size should be included in parentheses at the end of the figure caption. See Figure \ref{vid:satellite} for an example.

\begin{video}
\begin{center}
{\includegraphics[height=5cm]{satellite.eps}}
\\
\end{center}
\caption{\label{vid:satellite}This satellite is a still image from Video 1 (Video 1, MPEG, 2.5 MB).}
\end{video}

\appendix    % this command starts appendixes

\section{Miscellaneous Formatting Details}
\label{sect:misc}
At times it may be desired, for formatting reasons, to break a line without starting a new paragraph. In a LaTeX source file, a linebreak is created with \verb|\\|.


\subsection{Formatting Equations}
Equations may appear inline with the text, if they are simple, short, and not of major importance; for example, $\beta = b/r$.  Important equations appear on their own line.  Such equations are centered.  For example, ``The expression for the field of view is
\begin{equation}
\label{eq:fov}
2 a = \frac{(b + 1)}{3c} \, ,
\end{equation}
where $a$ is the ...''  Principal equations are numbered, with the equation number placed within parentheses and right justified.

Equations are considered to be part of a sentence and should be punctuated accordingly. In the above example, a comma appears after the equation because the next line is a subordinate clause. If the equation ends the sentence, a period should follow the equation. The line following an equation should not be indented unless it is meant to start a new paragraph. Indentation after an equation is avoided in LaTeX by not leaving a blank line between the equation and the subsequent text.

References to equations include the equation number in parentheses, for example, ``Equation~(\ref{eq:fov}) shows ...'' or ``Combining Eqs.~(2) and (3), we obtain...'' Note that the word ``Equation'' is spelled out if it begins a sentence, but is abbreviated as ``Eq.'' otherwise. Using a tilde in the LaTeX source file between two characters avoids unwanted line breaks, for example between ``Eq.'' and the following equation number..

\subsection{Formatting Theorems}

To include theorems in a formal way, the theorem identification should appear in a 10-point, bold font, left justified, and followed by a period.  The text of the theorem continues on the same line in normal, 10-pt. font, achieved in LaTeX using \verb|\footnotesize|.  For example,

\vspace{2ex}\noindent{\footnotesize\textbf{Theorem 1.} For any unbiased estimator...}

% \disclosures
\subsection*{Disclosures}
Conflicts of interest should be declared under a separate header. If the authors have no relevant financial interests in the manuscript and no other potential conflicts of interest to disclose, a statement to this effect should also be included in the manuscript.

\subsection* {Acknowledgments}
This unnumbered section is used to identify those who have aided the authors in understanding or accomplishing the work presented and to acknowledge sources of funding.

\subsection* {Data, Materials, and Code Availability}
As relevant, the availability of data, materials, and/or software code used in the research results reported in the manuscript may be declared in this section. (Note: this section is required for the \textit{Journal of Biomedical Optics} and \textit{Neurophotonics}.) Provide specific access information or restrictions for data, materials, and computer code (i.e., links to repository access addresses with guidance on commercial or public access).


%%%%% References %%%%%

\bibliography{report}   % bibliography data in report.bib
\bibliographystyle{spiejour}   % makes bibtex use spiejour.bst

%%%%% Biographies of authors %%%%%

\vspace{2ex}\noindent\textbf{First Author} is an assistant professor at the University of Optical Engineering. He received his BS and MS degrees in physics from the University of Optics in 1985 and 1987, respectively, and his PhD degree in optics from the Institute of Technology in 1991.  He is the author of more than 50 journal papers and has written three book chapters. His current research interests include optical interconnects, holography, and optoelectronic systems. He is a member of SPIE.

\vspace{1ex}
\noindent Biographies and photographs of the other authors are not available.

\listoffigures
\listoftables

\end{spacing}
\end{document}