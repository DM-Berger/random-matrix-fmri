\documentclass[10pt,a4paper]{article}
\usepackage[utf8]{inputenc}
\usepackage{csquotes}
\usepackage[british]{babel}
\usepackage{url}
\usepackage{graphicx}
\graphicspath{ {figures/} }
\usepackage{geometry}
\usepackage{amsmath}
\usepackage{graphicx}
% \usepackage[square,numbers]{natbib}
\usepackage[
    backend=biber,
    style=authoryear,
    natbib=true,
    sortlocale=en_US,
    url=false,
    doi=true,
    eprint=false
]{biblatex}
\addbibresource{report.bib}
\usepackage{fancyhdr}
\usepackage{color}
\definecolor{darkblue}{rgb}{0.,0.,0.4}
\definecolor{darkred}{rgb}{0.5,0.,0.}
\usepackage[pdftex,colorlinks=true,linkcolor=darkblue,citecolor=darkred,urlcolor=blue]{hyperref}

\begin{document}
% \bibliographystyle{apa}
\fancypagestyle{nofooter}{\fancyfoot{}}
\thispagestyle{nofooter}  %  https://tex.stackexchange.com/questions/136384/remove-footer-without-removing-the-header
\setlength{\headheight}{30.0pt}
\fancyhead[C]{\includegraphics[height=0.5in]{stfx_text.png}}
\fancyhead[L]{\includegraphics[height=0.5in]{crest.png}}
\fancyhead[R]{
    \sffamily
    \scriptsize
    4130 University Avenue \\
    B2G 2W5 \\
    Antigonish, Nova Scotia \\
    Canada
}
\normalsize

\title{Cover Letter}
% \begin{center}
% \includegraphics[width=4in]{stfx_logo.png}
% \end{center}


% \author[Derek Berger, Gurpreet Matharoo, Jacob Levman]{
%     Derek M. Berger\affil{1},
%     Gurpreet S. Matharoo\affil{2}
%     \and Jacob Levman\affil{3}
% }
% \affiliation{1}{Computer Science, St. Francis Xavier University, Antigonish, Canada, dberger@stfx.ca}
% \affiliation{2}{Physics, St. Francis Xavier University, Antigonish, Canada, gmatharo@stfx.ca}
% \affiliation{3}{Computer Science, St. Francis Xavier University, Antigonish, Canada, jlevman@stfx.ca}


% \keywords{random matrix theory, feature extraction, functional connectivity,
% functional magnetic resonance imaging, voxelwise connectivity, predictive analysis}

\noindent
\today, \\
Dear Dr. Giger, \\

We hope you will consider our manuscript \emph{Random Matrix Theory Tools for
the Predictive Analysis of Functional Magnetic Resonance Imaging Examinations}
for publication in SPIE Journal of Medical Imaging.

We confirm that this work is original and has not been published elsewhere nor
is it currently under consideration for publication elsewhere.

In this paper, we extract a number of eigenvalue-based features inspired by
Random Matrix Theory (RMT) from a wide variety of functional magnetic resonance
imaging (fMRI) data. We subsequently examine the cross-validated predictive
utility of those extracted features on a number of binary classification tasks
that can be constructed for each dataset. The predictive utility is examined
systematically, in a  multiverse-style analysis
\citep{steegenIncreasingTransparencyMultiverse2016}, which reveals the
sensitivity of results to various typical preprocessing decisions involved in
fMRI and RMT.

This paper is most unique in that it:

\begin{enumerate}
  \item employs RMT to generate reductions of the full voxelwise fMRI
        correlation matrix
  \item evaluates RMT-related metrics in a cross-validated predictive
        framework, instead of the more typical descriptive or explanatory
        frameworks
  \item systematically investigates the impact of the various methodological
        decisions involved in the findings
  \item makes all code used in the paper publicly available, including the release of a
        separate, general, open-source RMT
        library \citep{dm-bergerStfxecutablesEmpyricalRMTV12022} that will allow
        RMT metrics to be easily and reproducibly computed on a wide variety of
        data
\end{enumerate}

Please address all correspondence concerning this manuscript to me at \href{jlevman@stfx.ca}{jlevman@stfx.ca}.

Thank you for your consideration of this manuscript.




% apply mathematical tools from random matrix theory (RMT) to analyze

% , and release
% open-source software \citep{bergerStfxecutablesEmpyricalRMTPreliminary2020} to
% facilitate such analyses with a wide variety of data.

% Numerous fMRI studies examine correlations amongst regions of interest (ROI) in the brain. These
% ``functional connectivity'' analyses typically use ROIs or seed regions in order to simplify
% connectivity computations \citep{huettelFunctionalMagneticResonance2004}. Instead, we use
% statistics from RMT to summarize \emph{all} voxelwise correlations with no reductions. In
% addition, rather than simply reporting findings which are statistically significant, we report all
% analyses from a cross-validated predictive framework, where multiple machine learning models are
% compared against both each other and proper baselines (e.g. naive / dummy classifiers).

% We show that, across a variety of data and with fairly small data, and without any model tuning,
% that these RMT reductions have real predictive utility and potential. This suggests RMT could
% be a powerful and efficient method for summarizing functional connectivity information. As much of the
% research in network neuroscience involves fMRI, functional connectivity, and statistical physics, as
% per the \href{https://direct.mit.edu/netn/pages/submission-guidelines#aims}{Aims and Scope} of \emph{Network
% Neuroscience}, we believe our manuscript to be an excellent fit for submission to your journal.

% Should you consider to review our manuscript, we have suggested a number of potential reviewers on
% your submission website that we believe would be suitable due to expertise in RMT and similar
% interdisciplinary research. This paper is not under consideration at any other journa, although a
% poster of some preliminary results of our study was presented at the International Symposium for
% Magnetic Resonance in Medicine 2021 annual conference \citep{bergerOpenSourceRandom2021}. Our lab
% has REB ethics approval for conducting retrospective analyses (although we are unaware of the need,
% this being an analysis of anonymized public data). All authors approve the manuscript in its
% submitted form, and have no conflicts of interest, financial or otherwise.



% \begin{itemize}
% \item Sarika Jalan (sarika@iiti.ac.in) \\
% \small
% ~~~ Centre for Biosciences and Biomedical Engineering, IIT Indore India
% \normalsize
% \item Javeria Hashmi (javeria.hashmi@dal.ca) \\
% \small
% ~~~ Department of Anesthesia, Pain Management and Perioperative Medicine Dalhousie University, Canada
% \normalsize
% \item Nivedita Deo (ndeo@physics.du.ac.in) \\
% \small
% ~~~ Department of Physics and Astrophysics University of Delhi, India
% \normalsize
% \item Guillermo A. Cecchi (gcecchi@us.ibm.com) \\
% \small
% ~~~ Thomas J. Watson Research Center, Yorktown Heights, NY USA
% \normalsize
% \item Randy McIntosh (rmcintosh@research.baycrest.org) \\
% \small
% ~~~ Rotman Research Institute at Baycrest Health Sciences, Toronto, Canada
% \normalsize
% \end{itemize}


\bigskip

\noindent
Sincerely, \\

\small
\noindent
\textbf{Corresponding Author}\\
Jacob Levman, PhD (jlevman@stfx.ca) \\
% Canada Research Chair in Bioinformatics \\
Department of Computer Science \\
St. Francis Xavier University \\

\noindent
\textbf{Additional Contacts} [should the corresponding author be unavailable]\\
Derek Berger (dberger@stfx.ca), Department of Computer Science, St. Francis Xavier University\\
Gurpreet S. Matharoo (gmatharo@stfx.ca), Department of Physics, St. Francis Xavier University

% \printbibliography % biblatex
% \renewcommand{\bibsection}{}  % https://tex.stackexchange.com/questions/132646/how-to-remove-the-references-title
% \smallskip
% \bibliographystyle{mystyle}
% \bibliography{cover}  % natbib

\footnotesize
\begin{center}
\noindent\rule{6cm}{0.4pt}
\end{center}

\AtNextBibliography{\footnotesize}
\printbibliography[heading=none]

\end{document}