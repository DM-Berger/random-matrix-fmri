\documentclass[10pt, letterpaper]{article}
\usepackage[utf8]{inputenc}

\usepackage{geometry} % This package allows the editing of the page layout
\usepackage{amsmath}  % This package allows the use of a large range of mathematical formula, commands, and symbols
\usepackage{graphicx}  % This package allows the importing of images

\newcommand{\question}[2][]{
    \begin{flushleft}
    \textbf{Question #1}: \textit{#2}
    \end{flushleft}
}
\newcommand{\sol}{\textbf{Solution}:} %Use if you want a boldface solution line
\newcommand{\maketitletwo}[2][]{
    \begin{center}
        \Large{\textbf{Assignment #1}}

        Course Title

        \vspace{5pt}
        \normalsize{Matthew Frenkel}

        \today
        \vspace{15pt}
    \end{center}
}

\title{Random Matrix Theory Tools for the Analysis of Functional Magnetic Resonance Imaging Examinations}
\author{Derek M. Berger \and Gurpreet S. Matharoo \and Jacob Levman}
\date{\today}

\begin{document}
\maketitle


\begin{abstract}
    Previous studies have investigated the potential of using analytic techniques from Random Matrix
    Theory (RMT) to investigate magnetic resonance imaging (MRI) data. We assess the potential
    application of RMT-based features for the analysis of functional MRI (fMRI) across diverse
    datasets. As novel contributions, we (1) assess the potential for RMT-inspired, whole-brain
    features extracted from voxel-wise functional connectivity to contain information useful for
    classifying between various psychological processes, (2) assess these features’
    predictive—rather than explanatory—value, (3) investigate the effect of varying RMT analysis
    methods on the robustness of study findings, and (4) make general-purpose code publicly
    available for users to extract these features from a wide variety of data. We find preliminary
    evidence suggesting that RMT-inspired features may have unique potential in analyses of fMRI
    functional connectivity.

    \subparagraph{Index Terms:} feature extraction, functional connectivity, functional magnetic
    resonance imaging, random matrix theory, voxelwise connectivity

    \subparagraph{Authors}
    Derek Berger is with the department of Computer Science, St. Francis Xavier University, Antigonish
    NS B2G 2W5, Canada (e-mail: dberger@stfx.ca).

    Gurpreet Matharoo is with ACENET, and the Department of Physics, St. Francis Xavier University,
    Antigonish NS B2G 2W5, Canada (e-mail: gmatharo@stfx.ca).

    Jacob Levman is with the Department of Computer Science, St. Francis Xavier University,
    Antigonish NS B2G 2W5, Canada (e-mail: jlevman@stfx.ca).
\end{abstract}


\section{Introduction}

In functional magnetic resonance imaging (fMRI), changes in the blood-oxygenation-level-dependent
(BOLD) signals are related to neural activity. It is common to investigate statistical relationships
between the BOLD signals through functional connectivity analyses, where correlations between
collections of these signals are examined to infer connections between different voxels or regions
of interest (ROIs) within the brain.

Whether in the presence of experimental stimuli, or the relative absence, as in a resting state,
complex functional connectivity networks are ubiquitous [1]–[4]. This complexity suggests fMRI is a
candidate to be studied using Random Matrix Theory (RMT), a set of mathematical tools originally
developed some 50 years ago to solve complex problems in nuclear physics [5], [6]. In various
physical systems where nuclei can have different possible states, precisely modeling the entirety of
the interactions between nuclei is often intractable. For such large, complex systems, the
eigenvalues (or spectra) efficiently summarize the totality of the interactions between components
of the system, and RMT describes the expected behaviour of such eigenvalues. By analysing the
spectrum statistically, complex systems can be investigated by comparison to the universal
properties predicted by RMT.

\begin{sloppypar}
These universal properties have been observed in diverse phenomena. In small-scale systems, RMT
universalities have been observed in quantum chaotic systems, complex nuclei, atoms, molecules and
disordered mesoscopic systems [5]–[11], and at larger scales, RMT has been applied to atmospheric
physics [12], stock cross-correlations [13], social networks [14], random networks [15],
network-formation in liquids [16], [17] and amorphous clusters [18]–[21]. Within biological systems,
RMT has also been used to successfully model aspects of  amino acid functional relationships [22],
synchrony in epileptic seizures [23], and in protein-protein interactions both in different species
[24] and breast cancer [25].
\end{sloppypar}

In recent years, RMT analyses have been applied to study brain functioning.  The earliest  study
demonstrated that spectra of the correlations between electroencephalographic (EEG) signals closely
resemble those of the Gaussian Orthogonal Ensemble (GOE) [26].  Somewhat more recently, RMT was used
to evaluate the quality of whole brain features extracted from fMRI data [27], [28]. RMT has also
been used in diffusion MRI to aid in the selection of the number of components to employ in
principal-component reduction analysis and denoising [29]–[31].

Finally, RMT has been used in ROI-based fMRI functional-connectivity studies to investigate
differences between rest and task states [32], between subjects with and without ADHD [33], and
between pain and non-pain states [34]. Across these three studies, the spectra of resting or
low-attention states exhibited properties closest to the GOE.

These findings suggest that certain aspects of psychological processes might be characterized, in
part, by features computed from the eigenvalues of fMRI correlation matrices, and that these
features might vary in an interpretable manner across psychological processes. If this is the case,
RMT could have unique potential in characterizing the functioning of the human brain.

Here, we use a novel, voxel-based approach, and expand the applications of RMT to analyze fMRI scans
from  diverse datasets1. We extract RMT-inspired whole-brain features from voxelwise functional
connectivity data, and assess the predictive value of these features. We also examine in detail the
robustness of the RMT methodology to various common analytic choices.

\section{Materials and Methods}

\subsection{Datasets}

As our primary intention is to examine the potential utility of RMT-derived features across a wide
variety of data, and not to decisively put forward or refute any specific hypotheses relating RMT to
psychological processes, we examine a wide selection of fMRI datasets available on the OpenNeuro
platform [36].  This allows for a broad assessment of the potential of voxelwise RMT features.

However, since previous studies employing RMT for feature extraction [33], [34] have partially
interpreted their findings with respect to attentional processes, we have elected to use datasets
that involve attention in some way. This includes measures of attention or vigilance, or comparisons
between a condition involving more attention and focus (e.g. a task state) or conditions which
should involve considerably less attention or focus (e.g. resting-state).

\begin{center}
\texttt{\textbf{INSERT TABLE 1 HERE}}
\end{center}

Dataset scan parameters and acquisition details are summarized in Table I, and sample and subgroup
sizes are summarized in Table II. For all datasets, we use the raw (unpreprocessed) scans, and
perform our own preprocessing (detailed later). Given the complexity of these datasets, we highlight
below only those aspects relevant to the current paper.

\subsubsection{Dataset 1}
Dataset 1 (referred to in Fig. and tables with the label "PSYCH") has detailed and extensive
information available in  [37]. Subjects completed a battery of psychological measurements, and
various resting-state fMRI (rs-fMRI) images were obtained. Only the whole-brain 7-Tesla rs-fMRI
scans were used for the current paper, to make the results comparable to the other included
datasets. For the purpose of keeping with the "attention" theme of the analyses in this paper, three
pairings of subgroups were constructed via z-score splits of the various sum-scores of scale items
involving concepts closely related to attention.

\paragraph{Trait Attention}
The PANAS-X [38] is a 9-point Likert-type self-report measure that queries the extent to which
subjects agree particular affective terms have applied to them over a recent time period (e.g. a few
weeks). An ad hoc (i.e. face-valid) sum-score was constructed from the scale items: "attentive",
"alert", "interested", "concentrating", and reverse-scored items "sluggish", "tired", "sleepy",
"drowsy". Subjects with a z-score greater / less than zero on this sum-score were placed in the high
/ low "trait attention" category

\paragraph{Task Attention}
The Conjunctive Continuous Performance Task (CCPT [39]) is a behavioural measure of sustained
attention where subjects must react only if they see a red square among other different coloured
figures. For this paper, a sum score was constructed by subtracting the number of false positive and
false negative responses from the number of success responses, and subjects with a z-score greater /
less than zero on this scale were placed in high / low "task attention" category.

\paragraph{Vigilance}
The Mini New York cognition questionnaire assesses various elements of the subject's cognitive
state, e.g. "my thoughts involved other people", "I thought about something positive" (translated
from German) [40]. One question assesses alertness ("I was fully awake" / "war ich vollkommen wach")
and was used to create high / low "vigilance" conditions in the same manner as the previous two splits.

\subsubsection{Dataset 2}
Dataset 2 (referred to in Fig. and tables with the label "OSTEO") is detailed in [41]. Subjects
included healthy controls ("nopain" condition) and individuals with knee osteoarthritis ("pain"
condition) treated for 2 weeks with either placebo ("pain" condition) or duloxetine ("duloxetine"
condition). An additional grouping was also created with both the placebo and duloxetine data
("allpain" condition), allowing for six subgroup comparisons. All subjects were imaged with
whole-brain rs-fMRI. Scan parameters are documented in Table I. The use of rs-fMRI and RMT for
examining pain makes this dataset complementary to [34].

\subsubsection{Dataset 3}
Dataset 3  (referred to here with the label "REFLECT", due to the "reflective” nature of the
cognitive tasks) is described in full detail by DuPre, Luh and Spreng [42]. Both multi-echo rs- and
task-fMRI scans were conducted with scan parameters summarized in Table II. Tasks involved viewing
various affective images (see [43] for details) and performing tasks related to autobiographical
memory, prospection, or theory-of-mind reasoning. Continuing with the theme of broadly comparing
high versus low attention throughout this paper, the different tasks were simply treated generically
as "attending", in contrast to the non-attentive resting-state scans. This resulted in four task
state scans and one resting-state scan per subject.

In order to combine the multi-echo data into a single, usable matrix, two methods were employed. In
the "summed" condition, echoes were combined via summation, a method which has value in providing an
image with overall increased contrast [44]. In the "interleave" condition, echoes were interleaved
at each voxel: that is, if we denote the voxel's ith signal value from echo j as , then the
interleaved voxel of the new image has the time-series
\([t_{1,1}, t_{2,1}, t_{3,1}, \dots, t_{1,i}, t_{2,i}, t_{3,i}, \dots, t_{1,n}, t_{2,n}, t_{3,n}]\).

This creates a physically-meaningful 4D
image that contains the full, original multi-echo fMRI time-series data without any alterations. It
should be noted that the purpose of these combination methods is not to create a 4D volume that
allows optimal detection of the purely task-linked BOLD signal, but to ensure all scan information
was used with minimal additional analytic assumptions.

\subsubsection{Dataset 4}
Dataset 4 (referred to in Fig. and tables with the label "PARK" or "PARKINSONS") is documented and
described in detail in [45]. Scan parameters are documented in Table I. Subjects with non-demented
Parkinson’s Disease (PD) and healthy controls performed a number of repetitions of the Attention
Network Task (ANT) [46]. Subgroup details are available in Table I. As the ANT task demands high
attention, it is not as straightforward to create "high" or "low" attention groups. Instead, since
attention presumably is high and varies little in this dataset, and subgroups instead differ by
virtue of the presence or absence of PD, any results here serve as a useful comparison point for the
results from studies that should clearly involve attentional differences. This dataset also included
extensively preprocessed scans, which were included in analyses to examine the effects of
preprocessing on the extracted RMT features.

\subsubsection{Dataset 5}
Dataset 5 (referred to in Fig. and tables with the label "LEARN") is documented and described in
detail in [47]. In this study, both rs-fMRI and task fMRI scans were available for all subjects,
allowing for within-subject comparisons. The complex task involved multiple phases, wherein subjects
were exposed to named, black-and-white outlined satellite images related by family-resemblance of
the components (see Fig. 1 of [47]). After this learning task, subjects were tested on their memory
of the various learned features. Scans corresponding to both the initial learning and subsequent
testing phase were included in the high-attention "task" subgroup for the purposes of the analysis
of this paper. Resting state scans were included in the low-attention "rest" subgroup.

\subsection{Preprocessing}
Some consideration should be given to whether it is appropriate to apply standard preprocessing
steps before extracting RMT features, as preprocessing steps should be developed, tested, and
employed always with respect to a set of inferential goals and a specific analytic framework.
However, these preprocessing steps are increasingly recognized to have complex effects on fMRI
analyses [48]. For example, spatial smoothing has clear benefits when employing a general linear
model [49]–[51], but is less motivated in functional network analyses with small ROIs, where the
effect of smoothing can be considerable [50], [52], [53], with these effects interacting complexly
with factors like field strength and image resolution [54]. Similarly, image registration is usually
required to allow comparisons across (or even within) subjects, but the choice of registration
strategy can have dramatic effects on functional network analyses [55].

The RMT features extracted here are based on functional correlations rather than ROI activations,
and are completely insensitive to voxel location (spatial re-arrangement of an image’s voxels does
not change the eigenvalues of the correlation matrix). Thus RMT features summarize all voxel-wise
correlations in a whole-brain and location-free manner, which brings into question some of the usual
motivations for performing spatial smoothing, or registration to identify ROIs. In addition, if
image dimensions are identical between subjects, the RMT features (by virtue of their definition and
the unfolding procedure) already allow for direct comparisons. Thus, the usual motivations for
registration are also no longer clear when extracting RMT features.

Given these particularities, as well as the novelty of the RMT-based methods and the diversity of
the datasets, we elected to minimize the complexity of the preprocessing pipeline to ensure that
data processing is kept as standardized as possible. Ultimately, different preprocessing pipelines
for RMT fMRI analyses can be expected to have different benefits and shortcomings depending on the
question being investigated, and possibly also the specific RMT features being relied upon, but such
an investigation is beyond the scope of the current study.

For this paper’s analyses, brain extraction is necessary to limit analyses to the appropriate
voxels, motion correction is necessary to allow proper brain extraction, and both motion-correction
and slice-time correction simplify the interpretation of voxel-voxel correlations across voxel
locations. Whereas physiological noise regression and bias field correction would perhaps be
appropriate here, there is insufficient data to perform these steps for all datasets (Table II).
Thus, motion-correction was performed with FSL's MCFLIRT [56], slicetime correction with the default
method in FSL, and brain extraction was performed with BET [57]. Due to differences in the size of
the REFLECT data, the higher-resolution resting-state data was interpolated to have the same spatial
dimensions as the task data.


\section{Conclusion}



\paragraph{Declarations}
Manuscript submitted January 20, 2021. This work was supported by the  Natural Science and
Engineering Research Council of Canada's Canada Research Chair grant (grant number 231266) to JL,
Natural Science and Engineering Research Council of Canada Discovery Grant to JL, a Canada
Foundation for Innovation and Nova Scotia Research and Innovation Trust infrastructure grant
(R0176004) to JL, a St. Francis Xavier University research startup grant to JL (grant number
R0168020), and a St. Francis Xavier University UCR grant to JL.

\end{document}