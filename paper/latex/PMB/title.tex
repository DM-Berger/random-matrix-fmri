\documentclass[12pt]{iopart}
\usepackage{graphicx}
\usepackage{cleveref}
\graphicspath{ {figs/} }
% \documentclass[10pt, letterpaper]{article}
\usepackage[utf8]{inputenc}

\usepackage{geometry} % This package allows the editing of the page layout
\usepackage{amsmath}  % This package allows the use of a large range of mathematical formula, commands, and symbols
\usepackage{graphicx}  % This package allows the importing of images

\newcommand{\question}[2][]{
    \begin{flushleft}
    \textbf{Question #1}: \textit{#2}
    \end{flushleft}
}
\newcommand{\sol}{\textbf{Solution}:} %Use if you want a boldface solution line
\newcommand{\maketitletwo}[2][]{
    \begin{center}
        \Large{\textbf{Assignment #1}}

        Course Title

        \vspace{5pt}
        \normalsize{Matthew Frenkel}

        \today
        \vspace{15pt}
    \end{center}
}
% \documentclass{article}
% \usepackage[utf8]{inputenc}
% \usepackage{geometry}
% \usepackage{amsmath}
% \usepackage{graphicx}
% \usepackage[style=apa]{biblatex}
% \usepackage{natbib}
% \usepackage[font=footnotesize,labelfont=bf]{caption}
% \addbibresource{rmt.bib}
% \bibliographystyle{plain}
% \bibliography{rmt}
% \bibstyle{harvard}
\begin{document}

\title[Random Matrix Theory for fMRI]{Random Matrix Theory Tools for the Analysis of Functional Magnetic Resonance Imaging Examinations}
\author{D M Berger$^1$, Gurpreet S. Matharoo$^2$, Jacob Levman$^3$}
\address{$^1$ Computer Science, St. Francis Xavier University, Antigonish, Canada}
\address{$^2$ Physics, St. Francis Xavier University, Antigonish, Canada}
\address{$^3$ Computer Science, St. Francis Xavier University, Antigonish, Canada}
\ead{jlevman@stfx.ca}

% \date{\today}


\begin{abstract}
    Previous studies have investigated the potential of using analytic techniques from Random Matrix
    Theory (RMT) to investigate magnetic resonance imaging (MRI) data. We assess the potential
    application of RMT-based features for the analysis of functional MRI (fMRI) across diverse
    datasets. As novel contributions, we (1) assess the potential for RMT-inspired, whole-brain
    features extracted from voxel-wise functional connectivity to contain information useful for
    classifying between various psychological processes, (2) assess these features’
    predictive—rather than explanatory—value, (3) investigate the effect of varying RMT analysis
    methods on the robustness of study findings, and (4) make code publicly available to extract
    these features from a wide variety of data. We find preliminary evidence suggesting that
    RMT-inspired features may have unique potential in analyses of fMRI functional connectivity.
\end{abstract}

\keywords{random matrix theory, feature extraction, functional connectivity,
functional magnetic resonance imaging, voxelwise connectivity, predictive analysis}

\submitto{\PMB}
\maketitle

\end{document}
