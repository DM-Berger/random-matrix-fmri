\documentclass[NETN,manuscript]{stjour-new}
\articletype{Research}
\usepackage{graphicx}
\graphicspath{ {figs/} }
\usepackage{hyperref}
% \documentclass[10pt, letterpaper]{article}
\usepackage[utf8]{inputenc}

\usepackage{geometry} % This package allows the editing of the page layout
\usepackage{amsmath}  % This package allows the use of a large range of mathematical formula, commands, and symbols
\usepackage{graphicx}  % This package allows the importing of images

\newcommand{\question}[2][]{
    \begin{flushleft}
    \textbf{Question #1}: \textit{#2}
    \end{flushleft}
}
\newcommand{\sol}{\textbf{Solution}:} %Use if you want a boldface solution line
\newcommand{\maketitletwo}[2][]{
    \begin{center}
        \Large{\textbf{Assignment #1}}

        Course Title

        \vspace{5pt}
        \normalsize{Matthew Frenkel}

        \today
        \vspace{15pt}
    \end{center}
}
% \usepackage[utf8]{inputenc}
% \usepackage{geometry}
% \usepackage{amsmath}
% \usepackage{graphicx}
% \usepackage[style=apa]{biblatex}
% \usepackage{natbib}
% \usepackage[font=footnotesize,labelfont=bf]{caption}
% \addbibresource{rmt.bib}
% \bibliographystyle{plain}
% \bibliography{rmt}
\begin{document}

\title{Cover Letter}
\date{\today}

\author[Derek Berger, Gurpreet Matharoo, Jacob Levman]{
    Derek M. Berger\affil{1},
    Gurpreet S. Matharoo\affil{2}
    \and Jacob Levman\affil{3}
}
\affiliation{1}{Computer Science, St. Francis Xavier University, Antigonish, Canada, dberger@stfx.ca}
\affiliation{2}{Physics, St. Francis Xavier University, Antigonish, Canada, gmatharo@stfx.ca}
\affiliation{3}{Computer Science, St. Francis Xavier University, Antigonish, Canada, jlevman@stfx.ca}


% \keywords{random matrix theory, feature extraction, functional connectivity,
% functional magnetic resonance imaging, voxelwise connectivity, predictive analysis}

To the Editors of Network Neuroscience

We believe our manuscript \emph{Random Matrix Theory Tools for the Analysis of Functional Magnetic
Resonance Imaging Examinations} would be an excellent fit for Network Neuroscience as a
\emph{Research} article. Our use of random matrix theory (RMT) for the analysis of functional
connectivity in functional magnetic resonance imaging (fMRI) falls clearly within the \href{https://direct.mit.edu/netn/pages/submission-guidelines#aims}{Aims and Scope}
of \emph{Network Neuroscience}.


To the best of our knowledge, our voxelwise analysis of fMRI using RMT features is novel\(\mathbf{^{[1]}}\): the RMT
features computed here summarize the correlations of \emph{all} fMRI voxels (rather than those from
some seed voxels), but previous works (documented in the manuscript) have been ROI focused. We also
present results in a predictive framework using cross-validation on a wide variety of datasets,
which we believe is more rigorous than an approach based solely on statistical significance for a
single dataset. Given there are very few uses of RMT for fMRI analysis, we hope you will find this
manuscript to be worth your while.

Should you consider to review our manuscript, we believe the following reviewers would be suitable:

\begin{itemize}
\item Sarika Jalan (sarika@iiti.ac.in) \\
\small
~~~ Centre for Biosciences and Biomedical Engineering, IIT Indore India
\normalsize
\item Javeria Hashmi (javeria.hashmi@dal.ca) \\
\small
~~~ Department of Anesthesia, Pain Management and Perioperative Medicine Dalhousie University, Canada
\normalsize
\item Nivedita Deo (ndeo@physics.du.ac.in) \\
\small
~~~ Department of Physics and Astrophysics University of Delhi, India
\normalsize
\item Guillermo A. Cecchi (gcecchi@us.ibm.com) \\
\small
~~~ Thomas J. Watson Research Center, Yorktown Heights, NY USA
\normalsize
\item Randy McIntosh (rmcintosh@research.baycrest.org) \\
\small
~~~ Rotman Research Institute at Baycrest Health Sciences, Toronto, Canada
\normalsize
\end{itemize}

All authors approve the manuscript in its submitted form, and have no conflicts of interest,
financial or otherwise. [Names, affiliations, email addresses and funding information]

\textbf{[1]} A poster of some preliminary results of our study was presented at the ISMRM annual
conference \citep{bergerOpenSourceRandom2021}.

Sincerely, \\
Corresponding Author \\
Institution Title \\
Institution/Affiliation Name \\
Institution Address \\
Corresponding author e-mail address \\

~\\
\textbf{Additional Contact} [should the corresponding author be unavailable] \\
Institution Title \\
Institution/Affiliation Name \\
Institution Address \\
e-mail address \\

\bibliography{rmt}

\end{document}